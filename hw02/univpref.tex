\begin{homeworkProblem}{Universal Preference}
    
    Suppose that preferences in a stable matching system are universal: all $n$ jobs share the preferences $C_1 > C_2 > \dots >  C_n$ and all candidates share the preferences $J_1 > J_2 > \dots > J_n$.

    \begin{itemize}
        \item[A)] What pairing do we get from running the algorithm with jobs proposing? Prove that this happens for all $n$.
        \item[B)] What pairing do we get from running the algorithm with candidates proposing? Justify your answer.
        \item[C)] What does this tell us about the number of stable pairings? Justify your answer.
    \end{itemize}

    \part
    
    The pairings we get from running the stable matching algorithm with jobs proposing is one where, for an index $i$, candidate $C_i$ is paired with job $J_i$, i.e., the pairings we get are $\{ (C_1, J_1), (C_2, J_2), \dots, (C_n, J_n) \}$.
    \\ \\
    To prove that this happens for all $n$, I'll use the principle of induction to prove the \textit{base case} and the \textit{inductive step}.

    \begin{itemize}
        \item \textit{Base Case}: When $n=1$, the claim holds since there is only one possible pairing $(C_1, J_1)$.
        \item \textit{Inductive Hypothesis}: Assume that the pairing we get is $\{(C_1, J_1), (C_2, J_2), \dots, (C_k, J_k)\}$ for some value of $n=k$ where $k \in \mathbb{N}$.
        \item \textit{Inductive Step}: For $n = k+1$, we can show that the preferences and matching are the following:

        \begin{table}[h!]
        \centering
        \renewcommand{\arraystretch}{1.3} % Adjust row height
        \setlength{\tabcolsep}{10pt} % Adjust cell padding

        \begin{tabular}{cc}
            \begin{tabular}{|c|c|}
            \hline
            \textbf{Jobs} & \textbf{Candidates} \\ \hline
            $J_1$ & $C_1 > C_2 > \dots > C_k > C_{k+1}$ \\ \hline
            $J_2$ & $C_1 > C_2 > \dots > C_k > C_{k+1}$ \\ \hline
            $\vdots$ & $\vdots$ \\ \hline
            $J_k$ & $C_1 > C_2 > \dots > C_k > C_{k+1}$ \\ \hline
            $J_{k+1}$ & $C_1 > C_2 > \dots > C_k > C_{k+1}$ \\ \hline
            \end{tabular}
            &
            \begin{tabular}{|c|c|}
            \hline
            \textbf{Candidates} & \textbf{Jobs} \\ \hline
            $C_1$ & $J_1 > J_2 > \dots > J_k > J_{k+1}$ \\ \hline
            $C_2$ & $J_1 > J_2 > \dots > J_k > J_{k+1}$ \\ \hline
            $\vdots$ & $\vdots$ \\ \hline
            $C_k$ & $J_1 > J_2 > \dots > J_k > J_{k+1}$ \\ \hline
            $C_{k+1}$ & $J_1 > J_2 > \dots > J_k > J_{k+1}$ \\ \hline
            \end{tabular}
        \end{tabular}
        \end{table}

        The algorithm takes $k+1$ days to produce a stable matching. The resulting pairing is as follows. The circles indicate the job that a candidate picked on a given day (and rejected the rest).

        \begin{table}[h!]
        \centering
        \renewcommand{\arraystretch}{1.3} % Adjust row height
        \setlength{\tabcolsep}{10pt} % Adjust cell padding
        \begin{tabular}{|c||c|c|c|c|c|}
            \hline
            \textbf{Candidate} & \textbf{Day 1} & \textbf{Day 2} & \dots & \textbf{Day $k$} & \textbf{Day $k+1$} \\ \hline
            $C_1$ & $\Circled{1}, 2, \dots, k, k+1$ & \Circled{1} & \dots & \Circled{1} & \Circled{1} \\ \hline
            $C_2$ &  & $\Circled{2}, \dots, k, k+1$ & \dots & \Circled{2} & \Circled{2} \\ \hline
            \vdots & \vdots & \vdots & $\ddots$ & \vdots & \vdots \\ \hline
            $C_k$ &  &  &  & $\Circled{k}, k+1$ & \Circled{$k$} \\ \hline
            $C_{k+1}$ &  &  &  &  & \Circled{k+1} \\ \hline
        \end{tabular}
        \end{table}

        It takes 1 more day for the algorithm to terminate, and since candidate $C_{k+1}$ is least preferred by all jobs, it will be proposed to last. The job proposing will be job $J_{k+1}$, since it is least preferred by all candidates. Therefore, we get a pairing $\{(C_1, J_1), (C_2, J_2), \dots, (C_k, J_k), (C_{k+1}, J_{k+1})\}$. By induction, we have shown that we get a pairing $\{ (C_1, J_1), (C_2, J_2), \dots, (C_n, J_n) \}$ for all $n$.
        
    \end{itemize}

    \pagebreak
    
    \part
    
    We get the same pairing from running the algorithm with candidates proposing. Since the preferences are universal and mirrored from jobs to candidates, the candidates will propose and the jobs will reject in the exact same way as when the jobs proposed and the candidates rejected. Just replace ``candidates'' with ``jobs'' and vice versa, there must be no change. 
    \\ \\
    \part
    
    Because both the job-optimal and candidate-optimal stable matching algorithms produce the same pairing, there is only 1 stable pairing, and no intermediate stable pairings exist as both the worst and best cases are the same.
    
    
\end{homeworkProblem}