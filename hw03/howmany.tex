\begin{homeworkProblem}{How Many Solutions?}
    Consider the equation $ax \equiv b \pmod p$ for prime $p$. In the below three
    parts, when we discuss solutions, we mean a solution $x$ in the range $\{0, 
    1, \dots p-1\}$. In addition, include justification for your answers to all
    the subparts of this problem.

    \begin{itemize}
        \item[A)] For how many pairs $(a,b)$ does the equation have a unique solution?
        
        When $a \equiv 0 \pmod{p}$, for any $x$, the equation will have the same
        solution where $ax \equiv 0 \equiv b \pmod{p}$. Therefore, for the equation
        to produce a unique solution for $x$, $a \not\equiv 0 \pmod{p}$. There are 
        then $p-1$ options for $a$ and $p$ options for $b$ to form pairs, i.e.,
        there are $p(p-1)$ pairs $(a,b)$ for which the equation has unique solutions.
        
        \item[B)] For how many pairs $(a,b)$ does the equation have no solution?
        
        The equation has no solutions when $a \equiv 0 \pmod{p}$ but $b \not\equiv
        0 \pmod{p}$. Therefore, there is one option for $a$, 0, and $p-1$ not zero
        options for $b$, i.e., there are $p-1$ pairs $(a, b)$ for which the equation
        has no solutions. 

        \item[C)] For how many pairs $(a,b)$ does the equation have $p$ solutions?
        
        When both $a$ and $b$ are equivalent to 0 modulos $p$, there aren't any 
        unique solutions but there are $p$ solutions, since all elements in the 
        set $\{0, 1, \dots p-1\}$ times 0 will be equivalent to 0 modulos $p$.
        There is one pair, $(0, 0)$, for which the equation has $p$ solutions. 

    \end{itemize}

    Now, consider the equation $ax \equiv b \pmod{pq}$ for distinct primes $p,q$.
    In the below three parts, when we discuss solutions, we mean a solution $x$ 
    in the range $\{0, 1, \dots pq-1\}$.

    \begin{itemize}
        \item[D)] If $\gcd(a, pq) = p$, show that there exists a solution if and 
        only if $b = 0 \pmod p$. 

        If $b \equiv 0 \pmod{p}$, then $b \equiv 0 \pmod{pq}$ as well. In which 
        case $ax \equiv b \equiv 0 \pmod{pq}$ has a solution when $x = 0$. 

        If we only know that there exists a solution, we need to show that $b
        \equiv 0 \pmod{p}$. Since $a$ and $pq$ share $p$ as a greatest common 
        divisor, $b$ must also be a multiple of $p$ for $ax \equiv b \pmod{pq}$ 
        to hold. Therefore, $b \equiv 0 \pmod{p}$.


        \item[E)] If $\gcd(a, pq) = p$ and there is a solution $x$, show that there
        are exactly $p$ solutions. (Hint: consider how you can generate another solution
        $x + \_\_\_$)

        % From part D), it must then be the case, as per the biconditional, that $b$
        % is equivalent to 0 modulos $p$. Therefore, the equation $ax \equiv b \pmod{pq}$
        % becomes
        % \[
        %     \begin{split}
        %         ax \equiv b & \pmod{pq} \\
        %         k \cdot px \equiv l \cdot p & \pmod{pq} \\
        %         kx \equiv l & \pmod{pq}
        %     \end{split}
        % \]
        % Since both left- and right-hand sides of the equivalency became multiples 
        % of $p$, dividing by $p$ considers the equivalencies modulos $q$. 
        We can express $a$ as a multiple $k$ of $p$ in our equation to answer
        the question
        \[
            \begin{split}
                ax \equiv b & \pmod{pq} \\
                kp \cdot x \equiv b & \pmod{pq} \\
            \end{split} 
        \]

        To generate $x + \_\_\_$ solutions, we can add multiples $l$ of $q$ to 
        $x$, e.g. $kp(x + 2q)$, as each $klpq$ is equivalent to 0 modulos $q$. 
        We can only add up to the $p$ multiple of $q$ however, at which point 
        the solutions cycle and are identical to earlier ones.  

        \item[F)] For how many pairs $(a,b)$ are there exactly $p$ solutions? 
        
        The requirements from the last part are that $a$ and $pq$ share a greatest 
        common divisor of $p$, and form part D) that $b \equiv 0 \pmod{p}$. $b$
        can be expressed as a multiple of $p$, e.g. $kp$. $k$ must be less than
        $q$, otherwise $b$ becomes equivalent to $pq$ or out of range. Therefore,
        there are $q$ values $b$ can take on where $k \in \{ 0, 1, \dots, q-1 \}$.
        Similarly for $a$, it can be expressed as a multiple of $p$, e.g. $mp$.
        For $a$ to be divisible by $p$, but also not have $pq$ as a greatest
        common divisor and stay in the range, $m$ must be in the set $\{ 1, 2, \dots, q-1 \}$.
        Therefore, there are $q \cdot (q-1)$ pairs with exactly $p$ solutions.


    \end{itemize}
    
\end{homeworkProblem}