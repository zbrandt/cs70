\begin{homeworkProblem}{Touring Hypercube}

    An the lecture, you have seen that if $G$ is a hypercube of dimension $n$, 
    then
    \begin{itemize}
        \item The vertices of $G$ are the binary strings of length $n$.
        \item $u$ and $v$ are connected by an edge if they differ in exactly 
        one bit location.
    \end{itemize}
    
    A \emph{Hamiltonian tour} of a graph (with $n \geq 2$ vertices) is a tour that
    visits every vertex exactly once.

    \begin{itemize}
        \item[A)] Prove that a hypercube has an Eulerian tour if and only if $n$ 
        is even.
        \item[B)] Prove that every hypercube has a Hamiltonian tour. 
    \end{itemize}
    
    \part 

    An Eulerian tour is a sequence of edges, that starts and ends on the same vertex,
    in a graph that uses each edge exactly once. To prove that a hypercube has an 
    Eulerian tour if and only if $n$ is even, I will prove both directions of the
    biconditional. 
    \\ \\
    \textit{A hypercube has an Eulerian tour if $n$ is even.} To prove this, I 
    will show that a hypercube of even dimension only has vertices of even degree,
    which would imply that the hypercube has an Eulerian tour, since it is also
    connected. If $n$ is even, then $G$ is a hypercube of an even dimension, meaning
    the binary strings that are the vertices of $G$ are also of even length. Since
    each vertex is incident as many edges as their are one-bit-location differences
    in its binary string, if $n$ is even then there are also an even number of differences.
    This is because there are $n$ one-bit-location differences for a binary string,
    as each digit can either be zero or one. Therefore, since each vertex is incident
    to an even number of edges when $n$ is even, $G$ has an Eulerian tour.
    \\ \\
    \textit{$n$ is even if the hypercube has an Eulerian tour.} If a graph $G$ has
    an Eulerian tour, all its vertices must be of even degree. For a hypercube
    to have all its vertices of even degree, the binary strings must be of even
    length, i.e., $n$ must be even. If the binary strings were of odd length, that
    would mean vertices would be incident to an odd number of edges, and therefore 
    not have an Eulerian tour, leading to a contradiction.
    \\ \\
    \part 

    To prove that every hypercube has a Hamiltonian tour, I will use the principle
    of induction on $n$, the dimension of the hypercube.

    \begin{itemize}
        \item \textit{Base case}: For a hypercube $G$ to have at least two vertices,
        it must be of dimension $n=1$, as a binary string of length one has 1
        one-bit-location difference. $G$ then has a Hamiltonian tour, as traversing
        the one edge visits both vertices.
        \item \textit{Inductive hypothesis}: Assume that any hypercube of dimension
        $1 \leq n \leq k$ has a Hamiltonian tour where $k \in \mathbb{N}$. 
        \item \textit{Inductive step}: For $n=k+1$, we can show that the hypercube
        has a Hamiltonian tour. The $k+1$ dimensional hypercube is composed of two 
        $k$ dimensional hypercubes that, under the inductive hypothesis, each have
        a Hamiltonian tour. It is then possible to construct a consolidated Hamiltonian
        tour for the $k+1$ hypercube by removing the closing step of one of the
        tours and replacing it with one of the edges that crosses to the other $k$
        dimensional hypercube, following that tour backwards until the very last 
        step again, and then finally crossing over again to the starting vertex 
        to finish the Hamiltonian tour. Therefore, there exists a Hamiltonian tour for every
        hypercube.
    \end{itemize}


\end{homeworkProblem}