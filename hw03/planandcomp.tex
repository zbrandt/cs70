\begin{homeworkProblem}{Planarity and Graph Complements}

    Let $G = (V, E)$ be an undirected graph.  We define the complement of $G$ as $\overline{G} = (V, \overline{E})$ where $\overline{E} = \{(i,j) \mid i,j \in V, i \neq j\} - E$; that is, $\overline{G}$ has the same set of vertices as $G$, but an edge $e$ exists is $\overline{G}$ if and only if it does not exist in $G$.

    \begin{itemize}
        \item[A)] Suppose $G$ has $v$ vertices and $e$ edges.  How many edges does $\overline{G}$ have?
        \item[B)] Prove that for any graph with at least 13 vertices, $G$ being planar implies that $\overline{G}$ is non-planar.
        \item[C)] Now consider the converse of the previous part, i.e., for any graph $G$ with at least 13 vertices, if $\overline{G}$ is non-planar, then $G$ is planar. Construct a counterexample to show that the converse does not hold.
    \end{itemize}

    \textit{Hint: Recall that if a graph contains a copy of $K_5$, then it is non-planar. Can this fact be used to construct a counterexample?}
        
\end{homeworkProblem}