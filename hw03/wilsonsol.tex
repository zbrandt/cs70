\begin{homeworkProblem}{Wilson's Theorem}

    Wilson's Theorem states the following is true if and only if $p$ is prime:
    \[(p - 1)! \equiv -1 \pmod{p}.\]
    Prove both directions (it holds if AND only if $p$ is prime).
    \\ \\
    Hint for the if direction: Consider rearranging the terms in $(p - 1)! = 1 
    \cdot 2 \cdot \cdots \cdot (p - 1)$ to pair up terms with their inverses, when 
    possible. What terms are left unpaired?
    \\ \\
    Hint for the only if direction: If $p$ is composite, then it has some prime 
    factor $q$.  What can we say about $(p-1)! \pmod{q}$?
    \\ \\
    \solution 

    For the if direction, we can rearrange the terms in the equivalency to pair up 
    terms with their inverses. 
    \[
        \begin{split}
            (p - 1)! \equiv -1 & \pmod{p} \\
            1 \cdot 2 \cdot \dots \cdot (p -1 ) \equiv -1 & \pmod{p}
        \end{split}
    \]
    
    Since $p$ is prime, it's greatest common divisor with any other number is 1. 
    Therefore, each number in the product series $1, 2, \dots, (p-1)$ has a unique 
    multiplicative inverse modulo $p$. Since every prime number is odd (if there 
    existed an even prime number, it would be divisible by 2, and no longer be 
    prime), there are an even number of terms in the $(p-1)!$ product series. It 
    seems to be that, since there are an even number of terms, and each term has 
    a unique inverse, that the series is equivalent to 1. However, the term 1 is 
    its own inverse, and so is not paired up with any other term. Additionally,
    $p-1$ is also its own inverse, $(p-1) \cdot (p-1) = p^2 + 2p + 1 \equiv 1
    \pmod{p}$. Therefore, the product series of $(p-1)!$ is equivalent to $p-1$.
    And since $p$ is a multiple of $p$, this demonstrates the equivalency of $(p-1)!$
    and -1. 
    \[
        \begin{split}
            1 \cdot 2 \cdot \dots \cdot (p -1 ) \equiv -1 & \pmod{p} \\
            p-1 \equiv -1 & \pmod{p} \\
            -1 \equiv -1 & \pmod{p} \\
        \end{split}
    \]

    For the only if direction, assume that $p$ is not prime yet the equivalency 
    remains true. Therefore, if it is not prime, and greater than 1, $p$ is composite,
    and has some prime factor $q$. We then know that $(p-1)! \equiv 0 \pmod{q}$,
    since $q$ is less than $p$, $q$ is somewhere in the $(p-1)!$ product series,
    which can then be expressed as some multiple of $q$, i.e. $(p-1)! = kq$ for 
    some $k \in \mathbb{N}$. The original equivalency can be expressed as $(p-1)!
    = lp -1$. But since $p$ is composite, $(p-1)! = kq - 1$, which contradicts
    what we initially found, that $(p-1)!$ is a multiple of $q$ without any remainder.
    Therefore, $p$ cannot be composite and must be prime. 

    
\end{homeworkProblem}