\documentclass{article}

\usepackage{fancyhdr}
\usepackage{extramarks}
\usepackage{amsmath}
\usepackage{amsthm}
\usepackage{amsfonts}
\usepackage{tikz}
\usepackage[plain]{algorithm}
\usepackage{algpseudocode}

\usetikzlibrary{automata,positioning}

%
% Basic Document Settings
%

\topmargin=-0.45in
\evensidemargin=0in
\oddsidemargin=0in
\textwidth=6.5in
\textheight=9.0in
\headsep=0.25in

\linespread{1.1}

\pagestyle{fancy}
\lhead{\hmwkAuthorName}
\chead{\hmwkClass\ (\hmwkClassInstructor\ \hmwkClassTime): \hmwkTitle}
\rhead{\firstxmark}
\lfoot{\lastxmark}
\cfoot{\thepage}

\renewcommand\headrulewidth{0.4pt}
\renewcommand\footrulewidth{0.4pt}

\setlength\parindent{0pt}

%
% Create Problem Sections
%

\newcommand{\enterProblemHeader}[1]{
    \nobreak\extramarks{}{Problem \arabic{#1} continued on next page\ldots}\nobreak{}
    \nobreak\extramarks{Problem \arabic{#1} (continued)}{Problem \arabic{#1} continued on next page\ldots}\nobreak{}
}

\newcommand{\exitProblemHeader}[1]{
    \nobreak\extramarks{Problem \arabic{#1} (continued)}{Problem \arabic{#1} continued on next page\ldots}\nobreak{}
    \stepcounter{#1}
    \nobreak\extramarks{Problem \arabic{#1}}{}\nobreak{}
}

\setcounter{secnumdepth}{0}
\newcounter{partCounter}
\newcounter{homeworkProblemCounter}
\setcounter{homeworkProblemCounter}{1}
\nobreak\extramarks{Problem \arabic{homeworkProblemCounter}}{}\nobreak{}

%
% Homework Problem Environment
%
% This environment takes an optional argument. When given, it will adjust the
% problem counter. This is useful for when the problems given for your
% assignment aren't sequential. See the last 3 problems of this template for an
% example.
%
\newenvironment{homeworkProblem}[1][-1]{
    \ifnum#1>0
        \setcounter{homeworkProblemCounter}{#1}
    \fi
    \section{Problem \arabic{homeworkProblemCounter}}
    \setcounter{partCounter}{1}
    \enterProblemHeader{homeworkProblemCounter}
}{
    \exitProblemHeader{homeworkProblemCounter}
}

%
% Homework Details
%   - Title
%   - Due date
%   - Class
%   - Section/Time
%   - Instructor
%   - Author
%

\newcommand{\hmwkTitle}{Homework\ \#0}
\newcommand{\hmwkDueDate}{January 25, 2025}
\newcommand{\hmwkClass}{CS70}
\newcommand{\hmwkClassTime}{Section A}
\newcommand{\hmwkClassInstructor}{Professor Satish Rao}
\newcommand{\hmwkAuthorName}{\textbf{Zachary Brandt}}

%
% Title Page
%

\title{
    \vspace{2in}
    \textmd{\textbf{\hmwkClass:\ \hmwkTitle}}\\
    \normalsize\vspace{0.1in}\small{Due\ on\ \hmwkDueDate\ at 4:00pm}\\
    \vspace{0.1in}\large{\textit{\hmwkClassInstructor\ \hmwkClassTime}}
    \vspace{3in}
}

\author{\hmwkAuthorName}
\date{}

\renewcommand{\part}[1]{\textbf{\large Part \Alph{partCounter}}\stepcounter{partCounter}\\}

%
% Various Helper Commands
%

% Useful for algorithms
\newcommand{\alg}[1]{\textsc{\bfseries \footnotesize #1}}

% For derivatives
\newcommand{\deriv}[1]{\frac{\mathrm{d}}{\mathrm{d}x} (#1)}

% For partial derivatives
\newcommand{\pderiv}[2]{\frac{\partial}{\partial #1} (#2)}

% Integral dx
\newcommand{\dx}{\mathrm{d}x}

% Alias for the Solution section header
\newcommand{\solution}{\textbf{\large Solution}}

% Probability commands: Expectation, Variance, Covariance, Bias
\newcommand{\E}{\mathrm{E}}
\newcommand{\Var}{\mathrm{Var}}
\newcommand{\Cov}{\mathrm{Cov}}
\newcommand{\Bias}{\mathrm{Bias}}

\begin{document}

\maketitle

\pagebreak

\begin{homeworkProblem}{Administrivia}
    \begin{itemize}
        \item[(a)] Make sure you are on the course Ed (for Q\&A) and 
        Gradescope (for submitting homeworks, including this one). Find and 
        familiarize yourself with the course website. What is its 
        homepage’s URL?
        \item[(b)] Read the policies page on the course website.
            \begin{itemize}
                \item[(i)] What is the breakdown of how your grade is 
                calculated, for both the homework and the no-homework 
                option? 
                \item[(ii)] What is the attendance policy for discussions?
                \item[(iii)] When are homeworks released and when are they 
                due? 
                \item[(iv)] How many ``drops" do you get for homeworks? How 
                many mini-vitamins will contribute to your grade? 
                \item[(v)] When is the final exam?
                \item[(vi)] What percentage score is needed to earn full 
                credit on a homework?
            \end{itemize}
    \end{itemize}

    \solution

    \begin{itemize}
        \item[(a)] \href{http://www.eecs70.org/}{http://www.eecs70.org/}
        \item[(b)] Policies:
            \begin{itemize}
                \item[(i)] Homework and no-homework grading:
                    \begin{multicols}{2}
                        \begin{itemize}
                            \item Discussion attendance: 5\%
                            \item Mini-Vitamins: 5\% (top 13 scores)
                            \item Homeworks: 15\% (lowest 3 dropped)
                            \item Midterm: 30\%
                            \item Final: 45\%
                        \end{itemize}
                        
                        \columnbreak % Forces a column break, if needed.
                        
                        
                        \begin{itemize}
                            \item Discussion attendance: 5\%
                            \item Mini-Vitamins: 5\% (top 13 scores)
                            \item Midterm: 36\%
                            \item Final: 54\%
                        \end{itemize}
                    \end{multicols}
                    
                    
                \item[(ii)] You need to attend at least 13 discussions to recieve full credit.
                \item[(iii)] Homeworks are released on Sunday and due Saturday of the next week at 4:00 pm.
                \item[(iv)] You get three drops for your homework. Only the top 13 scores for the Mini-Vitamins contribute towards your grade.
                \item[(v)] May 16, 2025 from 7:00 pm until 10:00 pm
                \item[(vi)] 73\%
            \end{itemize}
    \end{itemize}

\end{homeworkProblem}

\pagebreak

\begin{homeworkProblem}{Course Policies}
    Go to the course website and read the course policies carefully. Leave 
    a followup on Ed if you have any questions. Are the following 
    situations violations of course policy? Write "Yes" or "No", and a 
    short explanation for each.

    \begin{itemize}
        \item[(a)] Alice and Bob work on a problem in a study group. They 
        write up a solution together and submit it, noting on their 
        submissions that they wrote up their homework answers together.
        \item[(b)] Carol goes to a homework party and listens to Dan 
        describe his approach to a problem on the board, taking notes in 
        the process. She writes up her homework submission from her notes, 
        crediting Dan.
        \item[(c)] Erin comes across a proof that is part of a homework 
        problem while studying course material. She reads it and then, 
        after she has understood it, writes her own solution using the same 
        approach. She submits the homework with a citation to the website. 
        \item[(d)] Frank is having trouble with his homework and asks Grace 
        for help. Grace lets Frank look at her written solution. Frank 
        copies it onto his notebook and uses the copy to write and submit 
        his homework, crediting Grace. 
        \item[(e)] Heidi has completed her homework using \LaTeX. Her 
        friend Irene has been working on a homework problem for hours, and 
        asks Heidi for help. Heidi sends Irene her PDF solution, and Irene 
        uses it to write her own solution with a citation to Heidi.
        \item[(f)] Joe found homework solutions before they were officially 
        released, and every time he got stuck, he looked at the solutions 
        for a hint. He then cited the solutions as part of his submission.
    \end{itemize}

    \textbf{Justification}

    \begin{itemize}
        \item[(a)] \textbf{Yes}, Alice and Bob must each write up their own 
        solution to the homework problems.
        \item[(b)] \textbf{No}, Carol wrote her own solution to the 
        homework problems and credited her source.
        \item[(c)] \textbf{No}, Erin wrote her own solution to the homework 
        problem and credited her source.
        \item[(d)] \textbf{Yes}, Frank must come up with his own solution 
        and not copy Grace's verbatim. 
        \item[(e)] \textbf{Yes}, Irene must come up with her own solution 
        and not copy Heidi's verbatim, even if it is a different medium.
        \item[(f)] \textbf{Yes}, using any kind of homework solutions on a current assignment is prohibited. 
    \end{itemize}
    
\end{homeworkProblem}

\begin{homeworkProblem}{Use of Ed}
    Ed is incredibly useful for Q\&A in such a large-scale class. We will 
    use Ed for all important announcements. You should check it frequently. 
    We also highly encourage you to use Ed to ask questions and answer 
    questions from your fellow students.

    \begin{itemize}
        \item[(a)] Read the Ed Etiquette section of the course policies and 
        explain what is wrong with the following hypothetical student 
        question: ``Can someone explain the proof of Theorem XYZ to me?" 
        (Assume Theorem XYZ is a complicated concept.)
        \item[(b)] When are the weekly posts released? Are they required 
        reading?
        \item[(c)] If you have a question or concern not directly related 
        to the course content, where should you direct it? 
    \end{itemize}

    \solution

    \begin{itemize}
        \item[(a)] Explaining the proof might give away the answer. 
        Instead, you should explain in a way so other students can 
        understand the problem, without spoiling the answer to them. Also, 
        it would violate the 5 minute test. 
        \item[(b)] The weekly posts are released on Mondays and are 
        required reading.
        \item[(c)] Questions or concerns not regarding course content but 
        rather policy should be directed to \href{mailto:sp25@eecs70.org}
        {sp25@eecs70.org}.
    \end{itemize}
    
\end{homeworkProblem}

\pagebreak

\begin{homeworkProblem}
    Suppose we would like to fit a straight line through the origin, i.e.,
    \(Y_i = \beta_1 x_i + e_i\) with \(i = 1, \ldots, n\), \(\E [e_i] = 0\),
    and \(\Var [e_i] = \sigma^2_e\) and \(\Cov[e_i, e_j] = 0, \forall i \neq
    j\).
    \\

    \part

    Find the least squares esimator for \(\hat{\beta_1}\) for the slope
    \(\beta_1\).
    \\

    \solution

    To find the least squares estimator, we should minimize our Residual Sum
    of Squares, RSS:

    \[
        \begin{split}
            RSS &= \sum_{i = 1}^{n} {(Y_i - \hat{Y_i})}^2
            \\
            &= \sum_{i = 1}^{n} {(Y_i - \hat{\beta_1} x_i)}^2
        \end{split}
    \]

    By taking the partial derivative in respect to \(\hat{\beta_1}\), we get:

    \[
        \pderiv{
            \hat{\beta_1}
        }{RSS}
        = -2 \sum_{i = 1}^{n} {x_i (Y_i - \hat{\beta_1} x_i)}
        = 0
    \]

    This gives us:

    \[
        \begin{split}
            \sum_{i = 1}^{n} {x_i (Y_i - \hat{\beta_1} x_i)}
            &= \sum_{i = 1}^{n} {x_i Y_i} - \sum_{i = 1}^{n} \hat{\beta_1} x_i^2
            \\
            &= \sum_{i = 1}^{n} {x_i Y_i} - \hat{\beta_1}\sum_{i = 1}^{n} x_i^2
        \end{split}
    \]

    Solving for \(\hat{\beta_1}\) gives the final estimator for \(\beta_1\):

    \[
        \begin{split}
            \hat{\beta_1}
            &= \frac{
                \sum {x_i Y_i}
            }{
                \sum x_i^2
            }
        \end{split}
    \]

    \pagebreak

    \part

    Calculate the bias and the variance for the estimated slope
    \(\hat{\beta_1}\).
    \\

    \solution

    For the bias, we need to calculate the expected value
    \(\E[\hat{\beta_1}]\):

    \[
        \begin{split}
            \E[\hat{\beta_1}]
            &= \E \left[ \frac{
                \sum {x_i Y_i}
            }{
                \sum x_i^2
            }\right]
            \\
            &= \frac{
                \sum {x_i \E[Y_i]}
            }{
                \sum x_i^2
            }
            \\
            &= \frac{
                \sum {x_i (\beta_1 x_i)}
            }{
                \sum x_i^2
            }
            \\
            &= \frac{
                \sum {x_i^2 \beta_1}
            }{
                \sum x_i^2
            }
            \\
            &= \beta_1 \frac{
                \sum {x_i^2 \beta_1}
            }{
                \sum x_i^2
            }
            \\
            &= \beta_1
        \end{split}
    \]

    Thus since our estimator's expected value is \(\beta_1\), we can conclude
    that the bias of our estimator is 0.
    \\

    For the variance:

    \[
        \begin{split}
            \Var[\hat{\beta_1}]
            &= \Var \left[ \frac{
                \sum {x_i Y_i}
            }{
                \sum x_i^2
            }\right]
            \\
            &=
            \frac{
                \sum {x_i^2}
            }{
                \sum x_i^2 \sum x_i^2
            } \Var[Y_i]
            \\
            &=
            \frac{
                \sum {x_i^2}
            }{
                \sum x_i^2 \sum x_i^2
            } \Var[Y_i]
            \\
            &=
            \frac{
                1
            }{
                \sum x_i^2
            } \Var[Y_i]
            \\
            &=
            \frac{
                1
            }{
                \sum x_i^2
            } \sigma^2
            \\
            &=
            \frac{
                \sigma^2
            }{
                \sum x_i^2
            }
        \end{split}
    \]

\end{homeworkProblem}

\pagebreak

\begin{homeworkProblem}{Propositional Practice}
    In parts (a)–(b), convert the English sentences into propositional 
    logic. In parts (c)–(d), convert the propositions into English. For 
    parts (b) and (d), use the notation $a | b$ to denote the statement ``a 
    divides b”, and use the notation $P(x)$ to denote the statement ``x is a 
    prime number.”

    \begin{itemize}
        \item[(a)] For every real number $k$, there is a unique real 
        solution to $x^3 = k$
        \item[(b)] If $p$ is a prime number, then for any two natural numbers $a$ and $b$, if $p$ doesn’t divide $a$ and $p$ divides $ab$, then $p$ divides $b$.
        \item[(c)] $(\forall x, y \in \mathbb{R})[(xy = 0) \implies ((x = 0) \vee (y = 0))] $
        \item[(d)] $\neg ((\exists y \in \mathbb{N})[(\forall x \in \mathbb{N})[(x > y) \implies (( x | y) \vee P(x))]])$
    \end{itemize}

    \solution
    
    \begin{itemize}
        \item[(a)] $(\forall x \in \mathbb{R})[(\exists k \in \mathbb{R})
        (x^3 = k)]$
        \item[(b)] $(\exists k \in \mathbb{Z})[(\exists p \in \mathbb{R})(p 
        = 2k+1)] \implies (\forall a, b \in \mathbb{N})[[(\neg p | a) 
        \wedge (p | ab)] \implies (p|b)]$
        \item[(c)] For all numbers $x$ and $y$ in the set of real numbers, if the product $xy$ is equal to zero, then $x$ equals zero or $y$ equals zero.
        \item[(d)] It is not the case that there exists a number $y$ in the set of real numbers, such that for all numbers $x$ in the set of real numbers, if $x$ is greater than $y$, then $x$ divides $y$ or $x$ is a prime number. 
    \end{itemize}
    
\end{homeworkProblem}

\pagebreak

\end{document}