\begin{homeworkProblem}{Propositional Practice}
    In parts (a)–(b), convert the English sentences into propositional 
    logic. In parts (c)–(d), convert the propositions into English. For 
    parts (b) and (d), use the notation $a | b$ to denote the statement ``a 
    divides b”, and use the notation $P(x)$ to denote the statement ``x is a 
    prime number.”

    \begin{itemize}
        \item[(a)] For every real number $k$, there is a unique real 
        solution to $x^3 = k$
        \item[(b)] If $p$ is a prime number, then for any two natural numbers $a$ and $b$, if $p$ doesn’t divide $a$ and $p$ divides $ab$, then $p$ divides $b$.
        \item[(c)] $(\forall x, y \in \mathbb{R})[(xy = 0) \implies ((x = 0) \vee (y = 0))] $
        \item[(d)] $\neg ((\exists y \in \mathbb{N})[(\forall x \in \mathbb{N})[(x > y) \implies (( y | x) \vee P(x))]])$
    \end{itemize}

    \solution
    
    \begin{itemize}
        \item[(a)] $(\forall k \in \mathbb{R})[(\exists !x \in \mathbb{R})
        (x^3 = k)]$
        \item[(b)] $(\forall p \in \mathbb{R})(P(p) \implies (\forall a, b \in \mathbb{N})[[\neg(p | a) 
        \wedge (p | ab)] \implies (p|b)])$
        \item[(c)] For all numbers $x$ and $y$ in the set of real numbers, 
        if the product $xy$ is equal to zero, then $x$ equals zero or $y$ 
        equals zero.
        \item[(d)] It is not the case that there exists a number $y$ in the 
        set of natural numbers, such that for all numbers $x$ in the set of 
        natural numbers, if $x$ is greater than $y$, then $y$ divides $x$ 
        or $x$ is a prime number. 
    \end{itemize}
    
\end{homeworkProblem}