\documentclass{article}

\usepackage{fancyhdr}
\usepackage{extramarks}
\usepackage{amsmath}
\usepackage{amsthm}
\usepackage{amsfonts}
\usepackage{tikz}
\usepackage[plain]{algorithm}
\usepackage{algpseudocode}
\usepackage{hyperref}
\usepackage{multicol}
\usepackage{truthtable}

\usetikzlibrary{automata,positioning}

%
% Basic Document Settings
%

\topmargin=-0.45in
\evensidemargin=0in
\oddsidemargin=0in
\textwidth=6.5in
\textheight=9.0in
\headsep=0.25in

\linespread{1.1}

\pagestyle{fancy}
\lhead{\hmwkAuthorName}
\chead{\hmwkClass\ (\hmwkClassInstructor): \hmwkTitle}
\rhead{\firstxmark}
\lfoot{\lastxmark}
\cfoot{\thepage}

\renewcommand\headrulewidth{0.4pt}
\renewcommand\footrulewidth{0.4pt}

\setlength\parindent{0pt}

%
% Create Problem Sections
%

\newcommand{\enterProblemHeader}[1]{
    \nobreak\extramarks{}{Problem \arabic{#1} continued on next page\ldots}\nobreak{}
    \nobreak\extramarks{Problem \arabic{#1} (continued)}{Problem \arabic{#1} continued on next page\ldots}\nobreak{}
}

\newcommand{\exitProblemHeader}[1]{
    \nobreak\extramarks{Problem \arabic{#1} (continued)}{Problem \arabic{#1} continued on next page\ldots}\nobreak{}
    \stepcounter{#1}
    \nobreak\extramarks{Problem \arabic{#1}}{}\nobreak{}
}

\setcounter{secnumdepth}{0}
\newcounter{partCounter}
\newcounter{homeworkProblemCounter}
\setcounter{homeworkProblemCounter}{1}
\nobreak\extramarks{Problem \arabic{homeworkProblemCounter}}{}\nobreak{}

%
% Homework Problem Environment
%
% This environment takes an optional argument. When given, it will adjust the
% problem counter. This is useful for when the problems given for your
% assignment aren't sequential. See the last 3 problems of this template for an
% example.
%
\newenvironment{homeworkProblem}[2][-1]{
    \ifnum#1>0
        \setcounter{homeworkProblemCounter}{#1}
    \fi
    \section{Problem \arabic{homeworkProblemCounter}: #2}
    \setcounter{partCounter}{1}
    \enterProblemHeader{homeworkProblemCounter}
}{
    \exitProblemHeader{homeworkProblemCounter}
}

%
% Homework Details
%   - Title
%   - Due date
%   - Class
%   - Section/Time
%   - Instructor
%   - Author
%

\newcommand{\hmwkTitle}{Homework\ \#1}
\newcommand{\hmwkDueDate}{February 1, 2025}
\newcommand{\hmwkClass}{Discrete Mathematics}
\newcommand{\hmwkClassTime}{Section A}
\newcommand{\hmwkClassInstructor}{Professor Satish Rao}
\newcommand{\hmwkAuthorName}{\textbf{Zachary Brandt}}
\newcommand{\hmwkAuthorEmail}{\href{mailto:zbrandt@berkeley.edu}{zbrandt@berkeley.edu}}

%
% Title Page
%

\title{
    \vspace{2in}
    \textmd{\textbf{\hmwkClass:\ \hmwkTitle}}\\
    \normalsize\vspace{0.1in}\small{Due\ on\ \hmwkDueDate\ at 4:00pm}\\
    \vspace{0.1in}\large{\textit{\hmwkClassInstructor}}
    \vspace{3in}
}

\author{\hmwkAuthorName \\ \hmwkAuthorEmail}
\date{}

\renewcommand{\part}[1]{\textbf{\large Part \Alph{partCounter}}\stepcounter{partCounter}\\}

%
% Various Helper Commands
%

% Useful for algorithms
\newcommand{\alg}[1]{\textsc{\bfseries \footnotesize #1}}

% For derivatives
\newcommand{\deriv}[1]{\frac{\mathrm{d}}{\mathrm{d}x} (#1)}

% For partial derivatives
\newcommand{\pderiv}[2]{\frac{\partial}{\partial #1} (#2)}

% Integral dx
\newcommand{\dx}{\mathrm{d}x}

% Alias for the Solution section header
\newcommand{\solution}{\textbf{\large Solution}}

% Probability commands: Expectation, Variance, Covariance, Bias
\newcommand{\E}{\mathrm{E}}
\newcommand{\Var}{\mathrm{Var}}
\newcommand{\Cov}{\mathrm{Cov}}
\newcommand{\Bias}{\mathrm{Bias}}

\begin{document}

\maketitle

\pagebreak

\begin{homeworkProblem}{Calculus Review}

    In the probability section of this course, you will be expected to compute 
    derivatives, integrals, and double integrals. This question contains a couple 
    examples of the kinds of calculus you will encounter

    \begin{itemize}
        \item[A)] Compute the following integral
            \[
                \begin{split}
                    \int_{0}^{\infty} \sin(t)e^{-t} \, \mathrm{d}t        
                \end{split}
            \]
        \item[B)] Compute the double integral  
            \[
                \begin{split}
                    \iint_{R} 2x+y \, \mathrm{d}A,
                \end{split}
            \]
            where $R$ is the region bounded by the lines $x=1, y = 0,$ and $y=x$.
    \end{itemize}
    
    \part

    \[
        \begin{split}
            \int_0^\infty \sin(t) e^{-t} \mathrm{d}t &= \left( -\sin(t)e^{-t} \right)_0^\infty - \int_0^\infty - \cos(t)e^{-t} \mathrm{d}t \\
            &= 0 + \int_0^\infty \cos(t)e^{-t} \mathrm{d}t \\
            &= \left( -\cos(t)e^{-t} \right)_0^\infty - \int_0^\infty \sin(t)e^{-t} \mathrm{d}t  \\
            2\int_0^\infty \sin(t) e^{-t} \mathrm{d}t &= - \left( -\cos(0)e^{0}\right) \\
            \int_0^\infty \sin(t) e^{-t} \mathrm{d}t &= \frac{1}{2}
        \end{split}
    \]
    
    \part

    \[
        \begin{split}
            \iint_R 2x+y\,\mathrm{d}A &= \int_0^1 \int_0^x 2x + y \, \mathrm{d}y \, \mathrm{d}x \\
            &= \int_0^1 \left( 2xy + \frac{1}{2} y^2\right)_0^x \, \mathrm{d}x \\
            &= \int_0^1 2x^2 + \frac{1}{2} x^2 \, \mathrm{d}x \\
            &= \left( \frac{2}{3} x^3 + \frac{1}{6} x^3 \right)_0^1 \\
            &= \frac{2}{3} + \frac{1}{6} \\
            &= \frac{5}{6}
        \end{split}
    \]
    
\end{homeworkProblem}

\pagebreak

\begin{homeworkProblem}{More Logical Equivalences}

    Evaluate whether the expressions on the left and right sides are equivalent in each part, and briefly justify your answers.

    \begin{itemize}
        \item[A)] $\forall x (P(x) \implies Q(x)) \overset{?}{\equiv} \forall x P(x) \implies \forall x Q(x)$
        \item[B)] $\neg (\exists x (P(x) \vee Q(x))) \overset{?}{\equiv} \forall x (\neg P(x) \wedge \neg Q(x))$
        \item[C)] $\forall x ((P(x) \implies Q(x)) \wedge Q(x)) \overset{?}{\equiv} \forall x P(x)$ 
    \end{itemize}

    \textbf{Justification}
    \\ \\
    \textbf{Part A} \\
    The expression $\forall x (P(x) \implies Q(x))$ is not logically equivalent to $\forall x P(x) \implies \forall x Q(x)$. Consider the case where there exists an $x_1$ where $P(x_1)$ is false and there exists a different $x_2$ where $Q(x_2)$ is false, but for all other $x$ the respective predicates are true. This would make the right-hand side expression vacuously true as $\forall x P(x)$ is false. However, there could be a case on the left-hand side where $P(x_1) \implies Q(x_1)$ is vacuously true while $P(x_2) \implies Q(x_2)$ is false, therefore making the entire left-hand side expression false. 
    \\ \\
    \textbf{Part B} \\
    This is an application of De Morgan's Laws, and the expressions are logically equivalent. Since $\neg \forall x R(x) \equiv \exists \neg R(x)$, consider $P(x) \vee Q(x)$ to be that predicate $R(x)$. Then the negation flips the existential quantifer of the left-hand side expression to match the universal quantifier of the right-hand side expression. We then have $\neg ((P(x) \vee Q(x))$, and since $\neg (P(x) \vee Q(x)) \equiv (\neg P(x) \wedge \neg Q(x))$, we arrive at the right-hand side expression from driving the negation through.
    \\ \\
    \textbf{Part C} \\
    The expression $\forall x ((P(x) \implies Q(x)) \wedge Q(x))$ is not logically equivalent to $\forall x P(x)$. Consider a case where the predicate $P(x)$ is true for all $x$ and for the predicate $Q(x)$ this is not true. Therefore, the right-hand side expression would be true, but there would exist an $x$ where the implication $P(x) \implies Q(x)$ would be false. This would make the left-hand side expression false. 
    
\end{homeworkProblem}

\pagebreak



\pagebreak



\pagebreak

\end{document}